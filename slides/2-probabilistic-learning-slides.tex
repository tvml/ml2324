\documentclass[9pt, aspectratio=169, sectionpages, handout]{beamer}
%\documentclass{beamer}
\usetheme[mode=light]{sthlmnord}
\usepackage{irbookslide2}
\usepackage{url}
\usepackage{newalg}
\usepackage{pstricks,pst-node,pst-plot,pst-tree}
\usepackage{graphicx}
 \usepackage{color}
 \usepackage[]{amsmath}
%\usepackage[landscape]{epsfig}
\usepackage{xkcdcols}

\usepackage[italian]{babel}
\selectlanguage{italian}
%%\usepackage[T1]{fontenc}
%\usepackage[utf8]{inputenc}
\usepackage{pgfplots}
\pgfplotsset{compat=1.14}
\usepackage{cancel}
\usepackage{math}
\usepackage{commands}
\usepackage{adg}
\usepackage{tikz}
\usetikzlibrary{arrows,decorations.pathmorphing,positioning,backgrounds,fit}
\usetikzlibrary{automata}
\usetikzlibrary{arrows,decorations.pathmorphing,positioning,backgrounds,fit}
\usetikzlibrary{arrows,decorations.pathmorphing,positioning,backgrounds,fit}
\usetikzlibrary{decorations,positioning,backgrounds,fit,shadows,automata,backgrounds,patterns,snakes}
\usetikzlibrary{calc,intersections,through,shapes,scopes,chains,matrix,shapes.arrows}
%
%\definecolor{pale}{RGB}{245,245,200}

\definecolor{pale}{RGB}{145,145,145}
\definecolor{node}{RGB}{220,220,220}
%\definecolor{observed}{RGB}{145,145,145}
\definecolor{observed}{RGB}{240,140,140}
\definecolor{box}{RGB}{240,240,240}
%\definecolor{param}{RGB}{176,200,233}
\definecolor{param}{RGB}{255,255,221}

\definecolor{ft}{RGB}{210,40,40}
\definecolor{bt}{RGB}{54,94,155}
\definecolor{nt}{RGB}{90,90,90}
\definecolor{mt}{RGB}{168,27,49}
\definecolor{at}{RGB}{170,80,80}
%
\usepackage{colortbl}
\usepackage{environ}


%%%%%%%%%%%%%%%%%%%%
\newenvironment<>{definizione}[1]{%
 \begin{block}{#1}}{\end{block}}
 
\NewEnviron{specialframe}[1]{%
\begingroup
\setbeamercolor{background canvas}{bg=\cnBlue}
\begin{frame}{#1}
\cGrey{\BODY}
  \end{frame}
\endgroup
}

% > > > CTAN Packages
%		the subfiles package is being used to compile individual slides
%		using the main tex file.  Individual slides can then be included
%		into multiple slide decks.
\usepackage{subfiles}

% > > > Custom Packages
\usepackage{mhomath}

% > > > Generate some Lorem Ipsum placeholder text.
\usepackage{lipsum}
\usepackage{commands}
% > > >	Image File Paths
% 		Here you can add one or more paths to where your images are being
%		stored.  This will allow you to include only the image file 
%		name when placing it into your document.
%\graphicspath{{path1},{path2},{path3}}
\graphicspath{{./assets/}} 

\newcommand{\snord}{\cnordTen{sthlm}\cnordEight{NORD}}

% > > > Document Information
\subtitle{Probabilistic learning}
\title{Machine learning}
\newcommand{\course}{Corso di Laurea Magistrale in Informatica}
\newcommand{\filename}{a.a. 2021-2022}
\author{Prof. Giorgio Gambosi}
\institute{Università di Roma Tor Vergata}
\newcommand{\othertext}{}
%\titlegraphic{nordtitlelogodark}
\titlegraphic{logo}

% > > > pdf customizations via hyperref (pkg installed by beamer)
\hypersetup{
colorlinks=true,
% You might want to disable color links for you final draft.
%colorlinks=false,
linkcolor={nordNine},
citecolor={nordNine},
urlcolor={nordNine}
}

\newcommand {\Prob} {\textrm {P}}
\newcommand {\Query} {\textrm {q}}
\newcommand {\Term} {\textrm {w}}
\renewcommand {\P} {\textrm {P}}
 \newcommand {\Inf} {{\textrm{ Inf}}}
 \newcommand {\walk} {\textrm {walk}}
 \newcommand{\probability}{\textrm {probability}}
 \newcommand{\relevance}{{\textrm{ relevance}}}
 \newcommand{\documents}{{\textrm{ document}}}
\newcommand {\weight} {{\textrm{ weight}}}
 \newcommand{\random}{\textrm {random process}}
%%%%%%%%%%% %%%%%%%%%%%%%%%%
%%				 						%
%%   Random variable definitions in formulas  		%
%%										%
%%%%%%%%%%%%%%%%%%%%%%%%%%%%
  \newcommand{\stress}[1]{ {{\bf #1}} }
%\newcommand{\stress}[1]{ {\textrm{\em #1}}}
  
 \newcommand {\doc} {\stress {d}}
 \newcommand {\Vocabulary} {\stress {V}}
\newcommand{\CollSize}{\stress {N}}
% \newcommand{\Freq}{\stress tf}
  \newcommand{\Div}{{\cal{D}}}
  \newcommand{\Bin}{{\cal{B}}}
\newcommand {\p} {\stress {p}}
\newcommand {\q} { {(1-p)}}
\newcommand {\ql} {\stress {q}}

 \newcommand {\Elite} {\stress {E_q}}
\newcommand {\Elitet} {\stress {E_ \Term}}
\newcommand {\length} { {l}}
\newcommand {\tf} { {tf}}
\newcommand {\tfn} {\stress {tfn}}
\newcommand {\tfq} {\stress {tfq}}
%\newcommand {\avg} {{\textrm{\bf avg}}\_{\bf\textrm length}}
\newcommand {\TF} {\stress {TF}}
\newcommand {\TFG} {\stress {n}}
\newcommand {\FreqTotColl} {\stress {TFC}}
\newcommand {\FreqTotCollSample} {\stress {TFC_E}}
\newcommand {\FreqTotCollElitet}  {\stress {TFC_\Elitet}}
\newcommand {\FreqTotCollElite} %{\stress {TFC_\Elitet}}
{\stress{\sum_{d'\in\Elitet} length(d')}}
 \renewcommand{\iff}{\mbox{$\Leftrightarrow$}}
   \newcommand{\TotFreq}[1]{\sum_\stress {t'}\stress {tf}\left (\stress{ t'} | #1\right)}
 \newcommand{\Collection}{\stress {D}}
 \newcommand{\IDFE}{   I(n$_{\textrm e}$)B2}
 \newcommand {\Prior}{\stress {P(\Term)}}
% \newcommand{\docfrequency}{\stress {n$_{ \Term}$}}
% \newcommand {\avg} {\stress{$\overline l$}}
\newcommand{\docfrequency}{\stress {n_{\Term}}}
 \newcommand {\df}{ \stress {\docfrequency}}
 \newcommand {\avg} {\stress{\overline l}}
 
 \newcommand{\accauno}{\stress {H1}}
\newcommand{\accadue}{\stress {H2}}
\newcommand{\accatre}{\stress {H3}}

%---------------------------------------------------------------
\definecolor{darkgreen1}{rgb}{0,0.6,0}
\definecolor{darkgreen}{rgb}{0,0.5,0}
\definecolor{lightpink}{rgb}{1,0.9,0.9}
%\definecolor{lightblue}{rgb}{0.8,0.95,1}
\definecolor{lightblue}{rgb}{0.8,1,1}
\definecolor{bluemare}{rgb}{0,0.5,1}
\definecolor{darkblue}{rgb}{0,0,0.8}
\definecolor{darkmagenta}{rgb}{0.5,0,0.5}
\definecolor{darkred}{rgb}{0.9,0.1,0}
\definecolor{giallino}{rgb}{1,1,0.8}
\definecolor{flashgreen}{rgb}{0,0.9,0.9}
\definecolor{violetto}{rgb}{0.9,0.9,1}
%---------------------------------------------------------------

\setbeamercolor{math text}{fg=xkcdDeepSeaBlue,bg=white}
\setbeamercolor{math text inlined}{parent=math text}
\setbeamercolor{math text displayed}{parent=math text}

\addtobeamertemplate{frametitle}{\donotcoloroutermaths}

%=-=-=-=-=-=-=-=-=-=-=-=-=-=-=-=-=-=-=-=-=-=-=-=-=-=-=-=-=-=-=-=-=-=-=-=-=-=-=-=
%
%    DOCUMENT BEGINS HERE 
%
%=-=-=-=-=-=-=-=-=-=-=-=-=-=-=-=-=-=-=-=-=-=-=-=-=-=-=-=-=-=-=-=-=-=-=-=-=-=-=-=
\begin{document}

%=-=-=-=-=-=-=-=-=-=-=-=-=-=-=-=-=-=-=-=-=-=-=-=-=-=-=-=-=-=-=-=-=-=-=-=-=-=-=-=
%   TITLE START   -=-=-=-=-=-=-=-=-=-=-=-=-=-=-=-=-=-=-=-=-=-=-=-=-=-=-=-=-=-=-=
\titlepage


\begin{frame}{Probabilistic approaches}
As done before, we assume that the observed dataset (features and target) has been derived by randomly sampling:
\begin{itemize}
\item $\mathcal{X}$ according to the probability distribution $p_{\mathcal{D}_1}(x)$ (usually the uniform distribution)
\item $\mathcal{Y}$ according to the conditional distribution $p_{\mathcal{D}_2}(y|x)$
\end{itemize}
\begin{enumerate}
\item we may then consider a class of possible conditional distributions $\mathcal{P}$ and
\item select (infer) the ``best'' conditional distribution $p^*\in\mathcal{P}$ from the available knowledge (that is, the dataset), according to some measure $q$ 
\item given any new item $x$, apply  $p^*(y|\x)$ to assign probabilities for each possible value of the corresponding target
\item an independent \alert{decision strategy} must be applied to $p^*(y|\x)$ to return a specific prediction $h(\x)$
\end{enumerate}
\end{frame}

\begin{frame}{Inferring a best distribution}
\begin{itemize}
\item how to define the class of possible conditional distributions $p(y|\x)$?
\begin{itemize}
\item usually, parametric approach: distributions defined by a common (arbitrary) structure and a set of parameters
\end{itemize}
\item what is a measure $q(p, \mathcal{T})$ of the quality of the distribution (given the dataset $\mathcal{T}=(\X,\t)$)?
\begin{itemize}
\item this is related to how a dataset generated by randomly sampling from $\mathcal{D}_1$ (usually uniform) and $\mathcal{D}_2$ could be similar to the available dataset $\mathcal{T}$
\end{itemize}
\end{itemize}
\end{frame}

\begin{frame}{A different approach}
Instead of finding a best distribution $p^*\in\mathcal{P}$ and use it to predict target probabilities as $p^*(y|\x)$ for any element $\x$, we could 
\begin{itemize}
\item consider for each possible conditional distribution $p\in\mathcal{P}$ its quality $q(p, \mathcal{T})$
\item compose all conditional distributions $p(y|\x)$ each weighted by its quality $q(p, \mathcal{T})$ (for example by means of a weighted averaging)
\item apply the resulting distribution 
\end{itemize}
\end{frame}


\begin{frame}{Different strategies}
Assume $q$ takes the form of a probability distribution (of probability distribution)
\begin{itemize}
\item first approach: take the modal value (the distribution of maximum quality) and apply it to perform predictions
\item second approach: compute the expectation of the distributions, wrt the probability distribution $q$ 
\end{itemize}
\end{frame}


\begin{frame}\frametitle{Inference of predictive distribution}
\begin{block}{Dataset}
We assume elements in $\mathcal{T}$ correspond to a set of $n$ samples, independently drawn from the same probability distribution (that is, they are \alert{independent and identically distributed}, i.i.d): they can be seen as $n$ realizations of a single random variable. 
\end{block}

We are interested in learning, starting from $\mathcal{T}$, a \alert{predictive distribution} $p(\x|\X)$ (or $p(\x,t|\X,\t)$) for any new element (or element-target pair). We may interpret this as the probability that, in a random sampling, the element actually returned is indeed $\x$ (or $\x,t$).

\begin{itemize}
\item in the case that $\mathcal{T}=\X=\{\x_1,\ldots,\x_n\}$, we are interested in deriving the probability distribution $p(\x|\X)$ of a new element, given the knowledge of the set $\X$ 
\item in the case that $\mathcal{T}=(\X,\t)=\{(\x_1,t_1),\ldots,(\x_n,t_n)\}$, we are interested in deriving the joint probability distribution  $p(\x,t|\X,\t)$ or, assuming $p(\x|\X,\t)$ uniform and thus also independent from $\X,\t$, the conditional distribution $p(t|\x,\X,\t)$, given the knowledge of the set of pairs $\X,\t$ 
\end{itemize}
\end{frame}


\begin{frame}\frametitle{Probabilistic models}
A \alert{probabilistic model} is a collection of probability distributions with the same structure, defined over the data domain. Probability distribution are instances of the probabilistic model and are characterized by 
the values assumed by a set of \alert{parameters}.
\begin{example}
In a bivariate gaussian probabilistic model, distributions are characterized by the values assumed by:
\begin{enumerate}
\item the mean $\Vmu=(\mu_1,\mu_2)$ 
\item the covariance matrix $\VSigma=\left(
\begin{array}{cc}
\sigma_{11} &\sigma_{12} \\
\sigma_{21} &\sigma_{22} 
\end{array}
\right)$
\end{enumerate}
where $\sigma_{12}=\sigma_{21}$
\end{example}
A probabilistic model could be
\begin{description}
\item[Parametric] if the set of parameters is given, finite, and independent from the data
\item[Non parametric] if the set of parameters is not given in advance, but derives from the data
\end{description}
\end{frame}




\begin{frame}\frametitle{Likelihood}
Given a dataset $\mathcal{T}$ and a probability distribution $p$ of parameters $\Vtheta$ defined on the same data domain, 
\begin{itemize}
\item the \alert{likelihood} of $\Vtheta$ wrt $\mathcal{T}$ is defined as
\[
L(\Vtheta|\mathcal{T}) = p(\mathcal{T}|\Vtheta)
\]
the probability of the dataset (that the dataset is generated) under distribution $p$ with parameters $\Vtheta$

\item while the probability $p(\mathcal{T}|\Vtheta)$ is considered as a function of $p(\mathcal{T}|\Vtheta)$ with $\Vtheta$ fixed, the likelihood $L(\Vtheta|\mathcal{T})$ is a function of $\Vtheta$ with $\mathcal{T}$ fixed

\item parameters $\Vtheta$ are considered as (independent) variables (\alert{frequentist interpretation} of probability) 
\end{itemize}
\end{frame}


\begin{frame}\frametitle{Likelihood}
\begin{itemize}
\item By assuming that elements in $\mathcal{T}$ are i.i.d., 
\begin{align*}
L(\Vtheta|\mathcal{T}) &=p(\X|\Vtheta)=\prod_{i=1}^np(\x_i|\Vtheta)&\hspace{-8.5cm}\text{in the first case}\\
L(\Vtheta|\mathcal{T}) &=p(\X,\t|\Vtheta)=\prod_{i=1}^np(\x_i,t_i|\Vtheta)=\prod_{i=1}^np(t_i|\x_i,\Vtheta)p(\x_i|\Vtheta)=p(\x|\Vtheta)\prod_{i=1}^np(t_i|\x_i,\Vtheta)\\
&=p(\x)\prod_{i=1}^np(t_i|\x_i,\Vtheta)\propto\prod_{i=1}^np(t_i|\x_i,\Vtheta)&\hspace{-2.5cm}\text{in the second case, assuming $p(\x|\Vtheta)$ uniform}
\end{align*}
%\end{block}
%\begin{block}<1->{Idea}
%Determine the parameter value that maximize the likelihood
%\[L(\Vtheta|\X)=p(\X|\Vtheta)=\prod_{i=1}^{N}p(\x_{i}|\Vtheta)\]
\end{itemize}
\end{frame}


\begin{frame}\frametitle{Maximum likelihood estimate}
\begin{block}<1->{Approach}

\alert{Frequentist} point of view: parameters are deterministic variables, whose value is unknown and must be  estimated.

%\end{block}
%\begin{block}<1->{Idea}
Determine the parameter value that maximize the likelihood
\[
\Vtheta^*=\argmax{\Vtheta}{L(\Vtheta|\mathcal{T})}=\argmax{\Vtheta}{p(\X|\Vtheta)}=\argmax{\Vtheta}{\prod_{i=1}^{n}p(\x_{i}|\Vtheta)}
\]
or 
\[
\Vtheta^*=\argmax{\Vtheta}{L(\Vtheta|\mathcal{T})}=\argmax{\Vtheta}{p(\X,\t|\Vtheta)}=\argmax{\Vtheta}{p(\x)\prod_{i=1}^{n}p(t_i|\x_{i},\Vtheta)}=\argmax{\Vtheta}{\prod_{i=1}^{n}p(t_i|\x_{i},\Vtheta)}
\]
\end{block}
\end{frame}

\begin{frame}\frametitle{Maximum likelihood estimate}

\begin{block}<1->{Log-likelihood}
\[l(\Vtheta|\mathcal{T})=\ln L(\Vtheta|\mathcal{T})%=\sum_{i=1}^{N}\ln p(\x_{i}|\Vtheta)
\]
is usually preferrable, since products are turned into sums, while $\Vtheta^*$ remains the same (since log is a monotonic function), that is
\[\argmax{\Vtheta}l(\Vtheta|\mathcal{T})=\argmax{\Vtheta}{L(\Vtheta|\mathcal{T})}\]
\end{block}
\begin{block}<1->{Estimate}
\[
\Vtheta^*_{ML}=\argmax{\Vtheta}{p(\X|\Vtheta)}=\argmax{\Vtheta}{\sum_{i=1}^{n}\ln p(\x_{i}|\Vtheta)}
\]
or 
\[
\Vtheta^*_{ML}=\argmax{\Vtheta}{p(\X,\t|\Vtheta)}=\argmax{\Vtheta}{\sum_{i=1}^{n}\ln p(t_i|\x_{i},\Vtheta)}
\]
\end{block}
\end{frame}

\begin{frame}\frametitle{Maximum likelihood estimate}
\begin{block}<1->{Solution}
Solve the system
\[
\frac{\partial l(\Vtheta|\mathcal{T})}{\partial\theta_{i}}=0\hspace{2cm}i=1,\ldots,d
\]
more concisely,
\[
\nabla_{\Vtheta}l(\Vtheta|\mathcal{T})=%\left[
%     \begin{array}{c}
%          \frac{\partial l(\Vtheta|\X)}{\partial\theta_{1}}  \\
%          \vdots  \\
%            \frac{\partial l(\Vtheta|\X)}{\partial\theta_{d}}  
%     \end{array}
%     \right]= \left[\begin{array}{c}
%          0   \\
%          \vdots   \\
%          0 
%     \end{array}
%     \right]=
\Zero
\]
\end{block}
\begin{block}<1->{Prediction}
Probability of a new observation $\x$:
\begin{align*}
p(\x|\X)&=\int_{\Vtheta} p(\x|\Vtheta)p(\Vtheta|\X)d\Vtheta\approx\int_{\Vtheta} p(\x|\Vtheta^*_{ML})p(\Vtheta|\X)d\Vtheta=p(\x|\Vtheta^*_{ML})\int_{\Vtheta}p(\Vtheta|\X)d\Vtheta=p(\x|\Vtheta^*_{ML})
\end{align*}
Predictive distribution $t|\x$:
\begin{align*}
p(t|\x,\X,\t)&=\int_{\Vtheta} p(t|\x,\Vtheta)p(\Vtheta|\X,\t)d\Vtheta\approx\int_{\Vtheta} p(t|\x, \Vtheta^*_{ML})p(\Vtheta|\X)d\Vtheta=p(\x|\Vtheta^*_{ML})\int_{\Vtheta}p(\Vtheta|\X,\t)d\Vtheta=p(t|\x,\Vtheta^*_{ML})
\end{align*}
\end{block}

\end{frame}

\begin{frame}\frametitle{Maximum likelihood estimate}
\begin{example}
Collection $\X$ of $n$ binary events, modeled through a Bernoulli distribution with unknown parameter $\phi$
\[
p(x|\phi)=\phi^{x}(1-\phi)^{1-x}
\]
Likelihood: $L(\phi|\X)=\prod_{i=1}^{n}\phi^{x_{i}}(1-\phi)^{1-x_{i}}$

\bigskip
Log-likelihood: $l(\phi|\X)=\sum_{i=1}^{n}\left(x_{i}\ln\phi+(1-x_{i})\ln(1-\phi)\right)=n_{1}\ln\phi+n_{0}\ln(1-\phi)$

\medskip
where $n_{0}$ ($n_{1}$) is the number of events $x\in\X$ equal to 0 (1)
\[
\frac{\partial l(\phi|\X)}{\partial\phi}=\frac{n_{1}}{\phi}-\frac{n_{0}}{1-\phi}=0\hspace{1cm}\implies\hspace{1cm}{\phi^*}_{ML}=\frac{n_{1}}{n_{0}+n_{1}}=\frac{n_{1}}{n}
\]
\end{example}
\end{frame}


\begin{frame}\frametitle{Maximum likelihood estimate}
\begin{example}
Linear regression: collection $\X,\t$ of value-target pairs, modeled as $p(\x,t)=p(\x)p(t|\x, \w, \sigma^2)$, with $\w\in\Real^d$, $w_0\in\Real$:
\begin{itemize}
\item $p(\x)$ uniform
\item $p(t|\x,\w,\sigma^2)=\mathcal{N}(\w^T\x+w_0, 1/\beta)$ ($\beta$, the inverse of the variance, is the \alert{precision})
\end{itemize}
Likelihood: 
\[
L(\t|\X,\w,w_0,\beta)=\prod_{i=1}^{n}p(t_i|\x_i,\w,w_0,\beta)=\prod_{i=1}^{n}\mathcal{N}(\w^T\x_i+w_0, \beta)
\]

\bigskip
Log-likelihood: 
\begin{align*}
l(\t|\X,\w,w_0,\beta)=
&= -\frac{\beta}{2}\sum_{i=1}^{n}(\w^T\x_i+w_0-t_i)^2+\frac{n}{2}\ln\beta-\frac{n}{2}\ln (2\pi)
\end{align*}

\end{example}
\end{frame}

\begin{frame}\frametitle{Maximum likelihood estimate}
\begin{example}

\vspace{-.5cm}
\begin{align*}
\frac{\partial}{\partial w_k}l(\t|\X,\w,w_0,\beta)&=-\frac{\beta}{2}\sum_{i=1}^{n}(\w^T\x_i+w_0-t_i)x_{ik}\hspace{1cm}k=1,\ldots,d\\
\frac{\partial}{\partial w_0}l(\t|\X,\w,w_0,\beta)&=-\frac{\beta}{2}\sum_{i=1}^{n}(\w^T\x_i+w_0-t_i)\\
\frac{\partial}{\partial\beta}l(\t|\X,\w,w_0,\beta)&=-\frac{1}{2}\sum_{i=1}^{n}(\w^T\x_i+w_0-t_i)^2+\frac{n}{2\beta}
\end{align*}
The ML estimation for $\w, w_0$ (linear regression coefficients) is obtained as the solution of  the $(d+1,d+1)$ linear system 
\begin{align*}
\sum_{i=1}^{n}(\w^T\x_i+w_0-t_i)x_{ik}&=0\hspace{1cm}k=1,\ldots,d\\
\sum_{i=1}^{n}(\w^T\x_i+w_0-t_i)&=0
\end{align*}
The ML estimation for $\beta$ is obtained by
\[
-\frac{1}{2}\sum_{i=1}^{n}(\w^T\x_i+w_0-t_i)^2+\frac{n}{2\beta}=0 \hspace{1cm}\implies\hspace{1cm}\beta_{ML}=\left(\frac{1}{n}\sum_{i=1}^{n}(\w^T\x_i+w_0-t_i)^2\right)^{-1}
\]
\end{example}
\end{frame}


%\begin{frame}\frametitle{ML and latent variables}
%\begin{block}<1->{Hidden states}
%If the model includes latent variables,  the dataset $\X$ only specifies values for observed variables. Since the estimate can take into account only observed values, latent variables are \alert{marginalized}:
%\[
%{\hat{\Vtheta}}_{ML}=\argmax{\Vtheta}\sum_{i=1}^{N}\ln\int_{\Z} p(x_{i}|\Vtheta)d\Z
%\]
%where $\Z$ is the set of latent variables.
%\end{block}
%\begin{block}<1->{Deriving the maximum liklelihood}
%To compute ${\hat{\Vtheta}}_{ML}$, the usual approach is to set partial derivatives to 0:
%\[
%\frac{\partial}{\partial \theta_{j}}\sum_{i=1}^{N}\ln\int_{\Z} p(x_{i}|\Vtheta)d\Z=\sum_{i=1}^{N}\frac{\partial}{\partial \theta_{j}}\ln\int_{\Z} p(x_{i}|\Vtheta)d\Z=0
%\]
%This may turn out to be exceedingly complex.
%\end{block}
%\end{frame}

\begin{frame}\frametitle{ML and overfitting}
\begin{block}<1->{Overfitting}
Maximizing the likelihood of the observed dataset tends to result into an estimate too sensitive to the dataset values, hence into \alert{overfitting}. The obtained estimates are suitable to model observed data, but may be too specialized to be used to model different datasets.
\end{block}
\begin{block}<1->{Penalty functions}
An additional function $P(\Vtheta)$ can be introduced with the aim to limit overfitting and the overall complexity of the model.
This results in the following function to maximize
\[
C(\Vtheta|\X)=l(\Vtheta|\X)-P(\Vtheta)
\]
as a common case,  $P(\Vtheta)=\frac{\gamma }{2}\|\Vtheta\|^{2}$, with $\gamma$ a \alert{tuning} parameter.
\end{block}

\end{frame}



\begin{frame}\frametitle{Maximum a posteriori estimate}
\begin{block}<1->{Idea}
Inference through maximum a posteriori (MAP) is similar to ML, but $\Vtheta$ is now considered as a random variable (bayesian approach), whose distribution has to be derived from observations, also taking into account previous knowledge (prior distribution). The parameter value maximizing
\[
p(\Vtheta|\mathcal{T})=\frac{p(\mathcal{T}|\Vtheta)p(\Vtheta)}{p(\mathcal{T})}
\]
is computed.
\end{block}
\end{frame}

\begin{frame}\frametitle{Maximum a posteriori estimate}
\begin{block}<1->{Estimate}
\begin{align*}
\Vtheta^*_{MAP}&=\argmax{\Vtheta} p(\Vtheta|\mathcal{T})=\argmax{\Vtheta}p(\mathcal{T}|\Vtheta)p(\Vtheta)\\
&=\argmax{\Vtheta}L(\Vtheta|\mathcal{T})p(\Vtheta)=\argmax{\Vtheta}\left(l(\Vtheta|\mathcal{T})+\ln p(\Vtheta)\right)
\end{align*}
which results into
\[
\Vtheta^*_{MAP}=\argmax{\Vtheta}\left(\sum_{i=1}^{n}\ln p(\x_{i}|\Vtheta)+\ln  p(\Vtheta)\right)
\]
or 
\[
\Vtheta^*_{MAP}=\argmax{\Vtheta}\left(\sum_{i=1}^{n}\ln p(t_i|\x_{i},\Vtheta)+\ln  p(\Vtheta)\right)
\]
\end{block}
\end{frame}

\begin{frame}\frametitle{MAP and gaussian prior}
\begin{block}<1->{Hypothesis}
Assume $\Vtheta$ is distributed around the origin as a multivariate gaussian with uniform variance and null covariance.That is, 
\[
p(\Vtheta)\sim{\mathcal{N}}(\Vtheta|\Zero, \sigma^{2})=\frac{1}{(2\pi)^{d/2}\sigma^{d}}e^{-\frac{\|\Vtheta\|^{2}}{2\sigma^{2}}}\propto e^{-\frac{\|\Vtheta\|^{2}}{2\sigma^{2}}}%\Longleftrightarrow p(\theta_i)\sim{\cal N}(\theta|0, \sigma^{2}) \mbox{ for all }i=1,\ldots,d
\]
\end{block}
\begin{block}<1->{Inference}
From the hypothesis,
\begin{align*}
&\Vtheta^*_{MAP}=\argmax{\Vtheta} p(\Vtheta|\mathcal{T})=\argmax{\Vtheta}\left(l(\Vtheta|\mathcal{T})+\ln p(\Vtheta)\right)\\
&   =\argmax{\Vtheta}\left(l(\Vtheta|\mathcal{T})+\ln  e^{-\frac{\|\Vtheta\|^2}{2\sigma^2}}\right)=\argmax{\Vtheta}\left(l(\Vtheta|\mathcal{T})-\frac{\|\Vtheta\|^2}{2\sigma^2}\right)
\end{align*}
which is equal to the penalty function introduced before, if 
 $\gamma=\frac{1}{\sigma^{2}}$
\end{block}
\end{frame}

\begin{frame}\frametitle{MAP estimate}
\begin{example}
Collection $\X$ of $n$ binary events, modeled as a Bernoulli distribution with unknown parameter $\phi$. Initial knowledge of $\phi$ is modeled as a Beta distribution:
\[
p(\phi|\alpha,\beta)=\mbox{Beta}(\phi|\alpha,\beta)=\frac{\Gamma(\alpha+\beta)}{\Gamma(\alpha)\Gamma(\beta)}\phi^{\alpha-1}(1-\phi)^{\beta-1}
\]
Log-likelihood
\[
l(\phi|\X)=\sum_{i=1}^{n}\left(x_{i}\ln\phi+(1-x_{i})\ln(1-\phi)\right)=n_{1}\ln\phi+n_{0}\ln(1-\phi)
\]
%dove $n_{0}$ ($n_{1}$) \`e il numero di $x\in\X$ pari a 0 (1)
\[
\frac{\partial }{\partial\phi}\big(l(\phi|\X)+\ln\mbox{Beta}(\phi|\alpha,\beta)\big)=\frac{n_{1}}{\phi}-\frac{n_{0}}{1-\phi}+\frac{\alpha-1}{\phi}-\frac{\beta-1}{1-\phi}=0\hspace{.5cm}\implies
\]
\[
\phi^*_{MAP}=\frac{n_{1}+\alpha-1}{n+\alpha+\beta-2}
\]
\end{example}
\end{frame}


\begin{frame}{Note}
\begin{block}{Gamma function}
The function
\[
\Gamma(x)=\int_{0}^{\infty}t^{x-1}e^{-t}dt
\]
is an extension of the factorial to the real numbers field: in fact, for any integer $x$,
\[
\Gamma(x)=(x-1)!
\]
\end{block}
\end{frame}

%\begin{frame}{Bayesian estimation}
%\begin{block}<1->{Idea}
%The posterior distribution is derived
%\[
%p(\Vtheta|\X)=\frac{p(\X|\Vtheta)p(\Vtheta)}{p(\X)}=\frac{p(\X|\Vtheta)p(\Vtheta)}{\int_{\Vtheta} p(\X|\Vtheta)d\Vtheta}
%\]
%it is then possible, for example, to obtain mean $\me{\Vtheta|\X}$ and variance $\va{\Vtheta|\X}$
%\end{block}
%\begin{block}<1->{Prediction}
%\[
%p(x|\X)=\int_{\Vtheta}p(x|\Vtheta)p(\Vtheta|\X)d\Vtheta=\int_{\Vtheta}p(x|\Vtheta)\frac{p(\X|\Vtheta)p(\Vtheta)}{p(\X)}d\Vtheta
%\]
%\end{block}
%\end{frame}

\begin{frame}\frametitle{Applying bayesian inference}
\begin{block}<1->{Mode and mean}
Once the posterior distribution
\[
p(\Vtheta|\X)=\frac{p(\X|\Vtheta)p(\Vtheta)}{p(\X)}=\frac{p(\X|\Vtheta)p(\Vtheta)}{\int_{\Vtheta} p(\X|\Vtheta)d\Vtheta}
\]
is available, MAP estimate computes the most probable value (mode) $\Vtheta_{MAP}$ of the distribution. This may lead to inaccurate estimates, as in the figure below:
\centerline{
\includegraphics[scale=0.3,angle=0]{figs/Figure_2.pdf}
}
\end{block}
\end{frame}

\begin{frame}\frametitle{Applying bayesian inference}
\begin{block}<1->{Mode and mean}
A better estimation can be obtained by applying a fully bayesian approach and referring to the whole posterior distribution, for example by deriving the expectation of 
$\Vtheta$ w.r.t. $p(\Vtheta|\X)$,
\[
\Vtheta^{*}=\med{p(\Vtheta|\X)}{\Vtheta}=\int_{\Vtheta}\Vtheta p(\Vtheta|\X)d\Vtheta
\]
\end{block}
\end{frame}

\begin{frame}\frametitle{Bayesian estimate}
\begin{example}
Collection $\X$ of $n$ binary events, modeled as a Bernoulli distribution with unknown parameter $\phi$. Initial knowledge of $\phi$ is modeled as a Beta distribution:
\[
p(\phi|\alpha,\beta)=\mbox{Beta}(\phi|\alpha,\beta)=\frac{\Gamma(\alpha+\beta)}{\Gamma(\alpha)\Gamma(\beta)}\phi^{\alpha-1}(1-\phi)^{\beta-1}
\]
Posterior distribution
\begin{align*}
p(\phi|\X,\alpha,\beta)&=\frac{\prod_{i=1}^{N}\phi^{x_{i}}(1-\phi)^{1-x_{i}}p(\phi|\alpha,\beta)}{p(\X)}=\frac{\phi^{N_{1}+\alpha-1}(1-\phi)^{N_{0}+\beta-1}}{Z}
\end{align*}
%since $\int_{-\infty}^{+\infty}p(\phi|\X,\alpha,\beta)d\phi=1$, $Z$ must be equal to the normalizing coefficient of the distribution $\mbox{Beta}(\phi|\alpha+N_{1},\beta+N_{0})$. 
Hence,
\[
p(\phi|\X,\alpha,\beta)=\mbox{Beta}(\phi|\alpha+N_{1},\beta+N_{0})
\]
%\[
%\me{p(\phi|\X,\alpha,\beta)}=\frac{n_{1}+\alpha}{n+\alpha+\beta}
%\]
\end{example}
\end{frame}

%%\begin{frame}\frametitle{Bayesian inference}
%%\begin{block}<1->{Bayesian inference and mean square error}
%%Fully bayesian inference makes it possible to minimize the expected mean square error 
%%\begin{align*}
%%{\cal C}(\Vtheta_{est})=\med{p(\Vtheta|\X)}{\frac{1}{2}\|\Vtheta-\Vtheta_{est}\|^{2}}=\int_{\Vtheta}\frac{1}{2}\|\Vtheta-\Vtheta_{est}\|^{2}p(\Vtheta|\X)d\Vtheta
%%\end{align*}
%%in the estimation of $\Vtheta$. 
%%
%%The minimum of this cost can be derived as follows
%%\begin{align*}
%%0&=\frac{\partial {\cal C}(\Vtheta_{est})}{\partial\Vtheta_{est}}=\int_{\Vtheta}(\Vtheta-\Vtheta_{est})p(\Vtheta|\X)d\Vtheta\\
%%&=\int_{\Vtheta}\Vtheta p(\Vtheta|\X)d\Vtheta-\Vtheta_{est}\int_{\Vtheta}p(\Vtheta|\X)d\Vtheta=\Vtheta^{*}-\Vtheta_{est}
%%\end{align*}
%%thus, the minimum is obtained for $\Vtheta_{est}=\Vtheta^{*}$
%%\end{block}
%%\begin{block}<1->{Prediction}
%%According to a pure bayesian approach, the prediction can be derived as follows
%%\[
%%p(x|\X)=\int_{\Vtheta}p(x|\Vtheta)p(\Vtheta|\X)d\Vtheta=\int_{\Vtheta}p(x|\Vtheta)\frac{p(\X|\Vtheta)p(\Vtheta)}{p(\X)}d\Vtheta
%%\]
%%\end{block}
%%\end{frame}
%
%
\section{Model selection}

\begin{frame}\frametitle{Model selection}
%\begin{block}<1->{Comparing different models}
In the process described, a model (structure, hyper-parameter values) has been identified, in come way. How can we deal with this problem?

\medskip
This is performed through \alert{model selection}: identify, in a set of possible models, the one which we expect is best to represent the available data.

\medskip
Indeed, the one whose best (or a good) instantiation is best to represent the available data

\medskip
We need a way to compare models (not their instantiations), given the dataset
\end{frame}
%Instead of composing the predictions of all probabilistic models, select and apply the one which best suit wrt $\mathcal{T}$.
%Let $m_{1}, \ldots,m_{r}$ be a set of probabilistic models, each with its own set of parameters. Given a dataset $\mathcal{T}$, we wish to \alert{select} the probabilistic model which may best represent $\mathcal{T}$.

\begin{frame}\frametitle{Theoretical approach}
Use the posterior probability of each model, given the dataset
\[
p(m|\mathcal{T})=\frac{p(\mathcal{T}|m)p(m)}{p(\mathcal{T})}%\propto p(\mathcal{T}|m)p(m)
\]
Observe that:
\begin{itemize}
\item If we assume that no specific knowledge on probabilistic models is initially available, then the prior distribution is uniform.
\item The evidence $p(\mathcal{T})$ is a constant with respect to $m$
\end{itemize}
As a consequence, $p(m|\mathcal{T})\propto p(\mathcal{T}|m)$ and we may refer to the likelihood $ p(\mathcal{T}|m)$ in order to compare models
%\end{block}
\end{frame}

\begin{frame}\frametitle{Model comparison}
%\begin{block}<1->{Evidence}
The distribution $p(\mathcal{T}|m)$ is also the evidence of the dataset w.r.t. model parameters%. It can be obtained by marginalization of model parameters
\[
p(\mathcal{T}|m)=\int_{\Vtheta}p(\mathcal{T}|\Vtheta,m)p(\Vtheta|m)d\Vtheta
\]
%\end{block}
\end{frame}


\begin{frame}{Model selection in practice}
\begin{block}{Validation}
\begin{description}
  \item[Test set] Dataset is split into Training set (used for learning parameters) and Test set (used for measuring effectiveness). Good for large datasets: otherwise, small resulting training and test set (few data for fitting and validation)
   \item[Cross validation] Dataset partitioned into $K$ equal-sized sets. Iteratively, in $K$ phases, use one set as test set and the union of the other $K-1$ ones as training set ($K$-fold cross validation). Average validation measures.
   
   		As a particular case, iteratively leave one element out and use all other points as training set (Leave-one-out cross validation).
   		
   		Time consuming for large datasets and for models which are costly to fit.
\end{description}
\end{block}
\end{frame}

\begin{frame}{Model selection in practice}
\begin{block}{Information measures}
Faster methods to compare model effectiveness, based on computing measures which take into account data fitting and model complexity.
\begin{description}
  \item[Akaike Information Criterion (AIC)] Let $\Vtheta$ be the set of parameters of the model and let $\Vtheta_{ML}$ be their maximum likelihood estimate on the dataset $\X$. Then, 
  \[
  AIC=2|\Vtheta|-2\log p(\X|\Vtheta_{ML})=2|\Vtheta|-2\max{\Vtheta} l(\Vtheta|\X)
  \]
  lower values correspond to models to be preferred. 
   \item[Bayesian Information Criterion (BIC)] A variant of the above, defined as 
\begin{align*}
  BIC&=|\Vtheta|-\log |\X|2\log p(\X|\Vtheta_{ML})\\
  &=|\Vtheta|\log |\X|-2\max{\Vtheta}l(\Vtheta|\X)
\end{align*}
\end{description}

\end{block}
\end{frame}



\section{Example: learning in the dirichlet-multinomial model}

\begin{frame}{Language modeling}
A \alert{language model} is a (categorical) probability distribution on a vocabulary of terms (possibly, all words which occur in a large collection of documents). 

\begin{block}{Use}
A language model can be applied to predict the next term occurring in a text. The probability of occurrence of a term is related to its information content and is at the basis of a number of information retrieval techniques.
\end{block}

\begin{block}{Hypothesis}
It is assumed that the probability of occurrence of a term is independent from the preceding terms in a text (\alert{bag of words} model).
\end{block}

\begin{block}{Generative model}
Given a language model, it is possible to sample from the distribution to generate random documents statistically equivalent to the documents in the collection used to derive the model.
\end{block}
\end{frame}


\begin{frame}{Language model}
\begin{itemize}
  \item Let $\mathcal{T}=\{t_1,\ldots,t_n\}$ be the set of terms occurring in a given collection $\mathcal{C}$ of documents, after \alert{stop word} (common, non informative terms) removal and \alert{stemming} (reduction of words to their basic form). 
  \item For each $i=1,\ldots,n$ let $m_i$ be the multiplicity (number of occurrences) of term $t_i$ in $\mathcal{C}$
  \item A language model can be derived as a categorical distribution associated to a vector $\hat{\Vphi}=(\hat{\phi}_1,\ldots,\hat{\phi}_n)^T$ of probabilities: that is, 
\[
0\leq\hat{\phi}_i\leq 1 \hspace{3mm}i=1,\ldots,n \hspace{2cm}\sum_{i=1}^n\hat{\phi}_i=1
\]
where $\hat{\phi}_j=p(t_j|\mathcal{C})$
\end{itemize}
\end{frame}

\begin{frame}{Learning a language model by ML}
Applying maximum likelihood to derive term probabilities in the language model  results into setting
\[
\hat{\phi}_j=p(t_j|\mathcal{C})=\frac{m_j}{\sum_{k=1}^nm_k}=\frac{m_j}{N}
\]
where $N=\sum_{i=1}^nm_i$ is the overall number of occurrences in $\mathcal{C}$ after stopword removal.
\begin{block}{Smoothing}
According to this estimate, a term $t$ which never occurred in $\mathcal{C}$ has 
zero probability to be observed (black swan paradox).
Due to overfitting the model to the observed data, typical of ML estimation.

\medskip
Solution: assign small, non zero, probability to events (terms) not observed up to now. This is called \alert{smoothing}. 
\end{block}
\end{frame}


\begin{frame}{Bayesian learning of a language model}
We may apply the dirichlet-multinomial model: 
\begin{itemize}
  \item this implies defining a
Dirichlet prior $\mbox{Dir}(\Vphi|\Valpha)$, with 
  $\Valpha=(\alpha_1, \alpha_2,\ldots, \alpha_n)$
  that is,
\[
p(\phi_{1},\ldots,\phi_{n}|\Valpha)=\frac{1}{\Delta(\alpha_{1},\ldots,\alpha_{n})}\prod_{i=1}^{n}\phi_{i}^{\alpha_{i}-1}
\]
  \item  the posterior distribution of $\Vphi$ after $\mathcal{C}$ has been observed is then $\mbox{Dir}(\Vphi|\Valpha')$, where
  \[
  \Valpha'=(\alpha_1+m_1, \alpha_2+m_2,\ldots, \alpha_n+m_n)
  \]
  that is,
\[
p(\phi_{1},\ldots,\phi_{n}|\Valpha')=\frac{1}{\Delta(\alpha_{1}+m_1,\ldots,\alpha_{n}+m_n)}\prod_{i=1}^{n}\phi_{i}^{\alpha_{i}+m_i-1}
\]  
\end{itemize}
\end{frame}



\begin{frame}{Bayesian learning of a language model}
The language model $\hat{\Vphi}$ corresponds to the predictive posterior distribution 
\begin{align*}
\hat{\phi}_j=p(t_j|\mathcal{C},\Valpha)&=\int p(t_j|\Vphi)p(\Vphi|\mathcal{C},\Valpha)d\Vphi=\int \phi_j\mbox{Dir}(\Vphi|\Valpha')d\Vphi=\me{\phi_j}
\end{align*}
where $\me{\phi_j}$ is taken w.r.t. the distribution $\mbox{Dir}(\Vphi|\Valpha')$. Then, 
\[
\hat{\phi}_j=\frac{\alpha_j'}{\sum_{k=1}^n\alpha_k'}=\frac{\alpha_j+m_j}{\sum_{k=1}^n(\alpha_k+m_k)}=\frac{\alpha_j+m_j}{\alpha_0+N}
\]
The $\alpha_j$ term makes it impossible to obtain zero probabilities (\alert{Dirichlet smoothing}).

\medskip
Non informative prior: $\alpha_i=\alpha$ for all $i$, which results into 
\[
p(t_j|\mathcal{C},\Valpha)=\frac{m_j+\alpha}{\alpha V+N}
\]
where $V$ is the vocabulary size.
\end{frame}

\begin{frame}{Naive bayes classifiers}
A language model can be applied to derive document classifiers into two or more classes.

\begin{itemize}
  \item given two classes $C_1, C_2$, assume that, for any document $d$, the probabilities $p(C_1|d)$ and $p(C_2|d)$ are known: then, $d$ can be assigned to the class with higher probability
  \item how to derive $p(C_k|d)$ for any document, given a collection $\mathcal{C}_1$ of documents known to belong to $C_1$ and a similar collection $\mathcal{C}_2$ for $C_2$? 
Apply Bayes' rule:
\[
p(C_k|d)\propto p(d|C_k)p(C_k)
\]
the evidence $p(d)$ is the same for both classes, and can be ignored.
\item we have still the problem of computing $p(C_k)$ and $p(d|C_k)$ from $\mathcal{C}_1$ and $\mathcal{C}_2$
\end{itemize}
\end{frame}


\begin{frame}{Naive bayes classifiers}
\begin{block}{Computing $p(C_k)$}
The prior probabilities $p(C_k)$ ($k=1,2$) can be easily estimated from $\mathcal{C}_1, \mathcal{C}_2$: for example, by applying ML, we obtain 
\[
p(C_k)=\frac{|\mathcal{C}_1|}{|\mathcal{C}_1|+|\mathcal{C}_2|}
\]
  \end{block}
  
\begin{block}{Computing $p(d|C_k)$}
For what concerns the likelihoods $p(d|C_k)$ ($k=1,2$), we observe that $d$ can be seen, according to the bag of words assumption, as a multiset of $n_d$ terms 
\[
d=\{\overline{t}_{1},\overline{t}_{2},\ldots, \overline{t}_{n_d}\}
\]
By applying the product rule, it results
\begin{align*}
p(d|C_k)&=p(\overline{t}_{1},\ldots, \overline{t}_{n_d}|C_k)=p(\overline{t}_{1}|C_k)p(\overline{t}_{2}|\overline{t}_1,C_k)\cdots p(\overline{t}_{n_d}|\overline{t}_{1},\ldots, \overline{t}_{n_d-1},C_k)
\end{align*}
  \end{block}
\end{frame}


\begin{frame}{Naive bayes classifiers}
\begin{block}{The naive Bayes assumption}
Computing $p(d|C_k)$ is much easier if we assume that terms are pairwise conditionally independent, given the class $C_k$, that is, for $i,j=1.\ldots, n_d$ and $k=1,2$,
\[
p(\overline{t}_{i},\overline{t}_{j}|C_k)=p(\overline{t}_{i}|C_k)p(\overline{t}_{2}|C_k)
\]
as, a consequence,
\[
p(d|C_k)=\prod_{j=1}^{n_d}p(\overline{t}_j|C_k)
\]
  \end{block}
  
\begin{block}{Language models and NB classifiers}
The probabilities $p(\overline{t}_j|C_k)$ are available for all terms if language models have been derived for $C_1$ and $C_2$, respectively from documents in $\mathcal{C}_1$ and $\mathcal{C}_2$.
\end{block}
\end{frame}

\begin{frame}{Feature selection by mutual information}
\begin{block}{Feature selection}
The set of probabilities in a language model can be exploited to identify the most relevant terms for classification, that is terms whose presence or absence in a document best characterizes the class of the document.
\end{block}

\begin{block}{Mutual information}
To measure relevance, we can apply the set of mutual informations $\{I_1,\ldots, I_n\}$ %between the class distribution and the conditional class distribution given a term
\begin{align*}
I_j&=\sum_{k=1,2}p(t_j,C_k)\log\frac{p(t_j,C_k)}{p(t_j)p(C_k)}\\
&=\sum_{k=1,2}p(C_k|t_j)p(t_j)\log\frac{p(C_k|t_j)}{p(C_k)}=p(t_j)KL(p(C_k|t_j)||p(C_k))
\end{align*}
here, $KL$ is a measure of the amount of information on class distributions provided by the presence of $t_j$. This amount is weighted by the probability of occurrence of $t_j$.
\end{block}
\end{frame}


\begin{frame}{Feature selection by mutual information}
\begin{block}{Mutual information}
Since $p(t_j,C_k)=p(C_k|t_j)p(t_j)=p(t_j|C_k)p(C_k)$, 
$I_j$ can be estimated as
\begin{align*}
I_j&=p(t_j|C_1)p(C_1)\log\frac{p(t_j|C_1)}{p(t_j)}+p(t_j|C_2)p(C_2)\log\frac{p(t_j|C_2)}{p(t_j)}\\
&=\phi_{j1}\pi_1\log\frac{\phi_{j1}}{\phi_{j1}\pi_1+\phi_{j2}\pi_2}+\phi_{j2}\pi_2\log\frac{\phi_{j2}}{\phi_{j1}\pi_1+\phi_{j2}\pi_2}
\end{align*}
where $\phi_{jk}$ is the estimated probability of $t_j$ in documents of class $C_k$ and $\pi_k$ is the estimated probability of a document of class $C_k$ in the collection.
\end{block}

A selection of the most significant terms can be performed by selecting the set of terms with highest mutual information $I_j$.

\end{frame}


%\section{Bayesian model comparison}
%
%\begin{frame}\frametitle{Model averaging}
%\begin{block}{Marginalization to reduce overfitting}
%\begin{itemize}
%\item To avoid overfitting, we may apply marginalization of model parameters: this corresponds to averaging among all possible models
%\item Bayesian approach: use of probabilities to represent uncertainty in the choice of the model
%\item Set of $L$ models ${\mathcal{M}_i}$, $i=1,\ldots, L$, each a probability distribution over the observed data $\mathcal{T}=(\X,\t)$ (conditional $p(\t|\X)$ or joint $p(\X,\t)$)
%\item Prior uncertainty about the model represented through distribution $p(\mathcal{M}_i)$
%\item Observing the training set modifies the uncertainty to the posterior
%\[
%p(\mathcal{M}_i|\mathcal{T})\propto p(\mathcal{T}|\mathcal{M}_i)p(\mathcal{M}_i)
%\]
%\item $p(\mathcal{T}|\mathcal{M}_i)$ is called \alert{marginal likelihood} or \alert{model evidence}
%\item $\displaystyle\frac{p(\mathcal{T}|\mathcal{M}_i)}{p(\mathcal{T}|\mathcal{M}_j)}$ is the \alert{Bayes factor} for models $\mathcal{M}_i, \mathcal{M}_j$
%\end{itemize}
%\end{block}
%\end{frame}
%
%\begin{frame}\frametitle{Model averaging}
%\begin{block}{Prediction}
%\begin{itemize}
%\item Given the posterior among models, the predictive distribution can be obtained as a mixture distribution
%\[
%p(t|\x,\mathcal{T})=\sum_{i=1}^Lp(t|\x,\mathcal{M}_i,\mathcal{T})p(\mathcal{M}_i|\mathcal{T})
%\]
%this corresponds to a weighted average among predictions of single models, with weights given by their probabilities
%\end{itemize}
%
%\end{block}
%\end{frame}
%
%\begin{frame}\frametitle{Model evidence}
%\begin{block}{As an average}
%\begin{itemize}
%\item The evidence of a model can be expressed as an average among instances for all possible parameter values
%\[
%p(\mathcal{T}|\mathcal{M}_i)=\int p(\mathcal{T}|\w, \mathcal{M}_i)p(\w|\mathcal{M}_i)d\w
%\]
%probability of generating $\mathcal{T}$ from a model with parameters derived by sampling distribution $p(\w|\mathcal{M}_i)$
%\item normalization term in definition of posterior distribution of parameters
%\[
%p(\w|\mathcal{T},\mathcal{M}_i)=\frac{p(\mathcal{T}|\w,\mathcal{M}_i)p(\w|\mathcal{M}_i)}{p(\mathcal{T}|\mathcal{M}_i)}
%\]
%\end{itemize}
%\end{block}
%\end{frame}
%
%
%\begin{frame}\frametitle{Model evidence}
%\begin{block}{Insight}
%\begin{itemize}
%\item Assume a model $\mathcal{M}$ with one parameter $w$
%\item Assume the posterior $p(w|\mathcal{T}, \mathcal{M})\propto p(\mathcal{T}|w, \mathcal{M})p(w|\mathcal{M})$ is sharply peaked around $w_{MAP}$, with width $\Delta w_{pos}$, hence 
%\[
%\int p(\mathcal{T}|w, \mathcal{M})p(\w|\mathcal{M})dw\simeq p(\mathcal{T}|w_{MAP}, \mathcal{M})p(w_{MAP}|\mathcal{M})\Delta w_{pos}
%\]
%\item Assume also a flat prior $p(w|\mathcal{M})$ with width $\Delta w_{pri}$ (and uniform probability $\displaystyle\frac{1}{\Delta w_{pri}}$) : then, 
%\begin{align*}
%p(\mathcal{T}|\mathcal{M})&=\int p(\mathcal{T}|w, \mathcal{M})p(w|\mathcal{M})dw\\
%&\simeq p(\mathcal{T}|w_{MAP}, \mathcal{M})p(w_{MAP}|\mathcal{M})\Delta w_{pos}\simeq p(\mathcal{T}|w_{MAP},\mathcal{M})\frac{\Delta w_{pos}}{\Delta w_{pri}}
%\end{align*}
%\end{itemize}
%\end{block}
%\end{frame}
%
%\begin{frame}\frametitle{Model evidence}
%
%\centerline{
%\includegraphics[scale=.7]{figs/Figure3-12.pdf}
%}
%\end{frame}
%
%\begin{frame}\frametitle{Model evidence}
%Taking logs, 
%\[
%\log p(\mathcal{T}|\mathcal{M})\simeq \log p(\mathcal{T}|w_{MAP},\mathcal{M})+\log\frac{\Delta w_{pos}}{\Delta w_{pri}}
%\]
%\begin{itemize}
%\item The first term is the fit of data to the most probable parameter values
%\item The second term is negative ($\Delta w_{pos}<\Delta w_{pri}$) and it is a penalization related to the model complexity
%\begin{itemize}
%\item $\Delta w_{pos}$ very small: the parameter is finely tuned to data (even small differences in its value make the dataset unlikely). The second term is negative and large in module: the model is quite penalized
%\item $\Delta w_{pos}$ large: the parameter is only roughly tuned to data (the dataset has the same fit also for different parameter values). The second term is still negative, but small in module: the model has a small penalization
%\end{itemize}
%\end{itemize}
%\end{frame}
%
%\begin{frame}\frametitle{Model selection}
%Model with a set of $M$ parameters, assuming all parameters have the same ratio $\displaystyle\frac{\Delta w_{pos}}{\Delta w_{pri}}$,
%\[
%\log p(\mathcal{T}|\mathcal{M})\simeq \log p(\mathcal{T}|\w_{MAP},\mathcal{M})+M\log\frac{\Delta w_{pos}}{\Delta w_{pri}}
%\]
%
%\medskip
%Model complexities
%\begin{itemize}
%\item $\mathcal{M}_1$, low complexity: few datasets fitted ($\mathcal{D}_0$ does not fit)
%\item $\mathcal{M}_3$, high complexity: many datasets fitted, with low probability ($\mathcal{D}_0$ fits poorly)
%\item $\mathcal{M}_2$, intermediate complexity: some datasets fitted ($\mathcal{D}_0$ fits better)
%\end{itemize}
% 
%\centerline{
%\includegraphics[scale=.7]{figs/Figure3-13.pdf}
%}
%\end{frame}
%
%
%
%\begin{frame}\frametitle{Model selection}
%%\begin{block}<1->{Managing overfitting}
%%Selecting the model type with highest dataset evidence 
%%$p(\mathcal{T}|{\cal M}_{i})$ allows to deal with overfitting, since it favors models of intemediate complexity. 
%%\end{block}
%\begin{description}
%\item[Too simple model.]
%If ${\cal M}_{i}$ is very simple, it will justify a limited collection of datasets (low generality) and $p(\mathcal{D}|{\cal M}_{i})$ will assume significant values in a limited domain. Then, $p(\mathcal{D}_0|{\cal M}_{i})$ will most likely be small, and ${\cal M}_{i}$ will not be selected.
%
%\item[Too complex model.]
%If ${\cal M}_{i}$ is very complex, it will justify a large collection of datasets (high generality) and $p(\mathcal{D}|{\cal M}_{i})$ will assume significant values in a large domain. As a consequence,  such values will be small, since
%\[
%\int_{\mathcal{D}}p(\mathcal{D}|{\cal M}_{i})d\mathcal{D}=1
%\]
%Then, it is likely that $p(\mathcal{D}|{\cal M}_{i})$ will be small, and  ${\cal M}_{i}$ will not be selected.
%\end{description}
%\end{frame}
%
%
%



\end{document}